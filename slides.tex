% Copyright 2004 by Till Tantau <tantau@users.sourceforge.net>.
%
% In principle, this file can be redistributed and/or modified under
% the terms of the GNU Public License, version 2.
%
% However, this file is supposed to be a template to be modified
% for your own needs. For this reason, if you use this file as a
% template and not specifically distribute it as part of a another
% package/program, I grant the extra permission to freely copy and
% modify this file as you see fit and even to delete this copyright
% notice. 

\documentclass{beamer}
\usepackage[utf8]{inputenc}
\usepackage{CJKutf8}
\usepackage{graphicx} %插入图片的宏包
\usepackage{float} %设置图片浮动位置的宏包
\usepackage{subfigure} %插入多图时用子图显示的宏包
% There are many different themes available for Beamer. A comprehensive
% list with examples is given here:
% http://deic.uab.es/~iblanes/beamer_gallery/index_by_theme.html
% You can uncomment the themes below if you would like to use a different
% one:
%\usetheme{AnnArbor}
%\usetheme{Antibes}
%\usetheme{Bergen}
%\usetheme{Berkeley}
%\usetheme{Berlin}
%\usetheme{Boadilla}
%\usetheme{boxes}
%\usetheme{CambridgeUS}
%\usetheme{Copenhagen}
%\usetheme{Darmstadt}
%\usetheme{default}
%\usetheme{Frankfurt}
%\usetheme{Goettingen}
%\usetheme{Hannover}
%\usetheme{Ilmenau}
%\usetheme{JuanLesPins}
%\usetheme{Luebeck}

%\usetheme{Madrid}
%\usetheme{Malmoe}
%\usetheme{Marburg}
%\usetheme{Montpellier}
%\usetheme{PaloAlto}
%\usetheme{Pittsburgh}
\usetheme{Rochester}
%\usetheme{Singapore}
%\usetheme{Szeged}
%\usetheme{Warsaw}

\title{Final Project}
\subtitle
{
  音訊分類辨識\\
  及視覺化呈現
}
% - Use the \inst command only if there are several affiliations.
% - Keep it simple, no one is interested in your street address.

\date{2019/06/15}
% - Either use conference name or its abbreviation.
% - Not really informative to the audience, more for people (including
%   yourself) who are reading the slides online

\AtBeginSubsection[]
{
  \begin{frame}<beamer>{Outline}
    \tableofcontents[currentsection,currentsubsection]
  \end{frame}
}

% Let's get started
\begin{document}
\begin{CJK}{UTF8}{bsmi}
\begin{frame}
  \titlepage
\end{frame}

\begin{frame}{Outline}
  \tableofcontents
  % You might wish to add the option [pausesections]
\end{frame}

% Section and subsections will appear in the presentation overview
% and table of contents.
\section{Introduction}

\begin{frame}{Introduction}
\textbf{Introduction to our team }
 \vspace{0.5cm}
\begin{description}

\item 組員4人\\
1051526 陳昱安\\
1051424 唐岳\\
1051423 張藝憲\\
1050758 劉紋琦\\
\vspace{0.5cm}
\end{description}
\textbf{Problem trying to solve }
\begin{description}
 \vspace{0.5cm}
\item 城市中有許多種類的聲音,如車聲,喇叭聲, \\
施工噪音等等。\\
我們希望能藉由程式識別這些聲音,\\
拓展至更多生活面的應用。
\end{description}
\end{frame}

\section{Methodology}
\begin{frame}{Methodology}{Input of model }
  \begin{itemize}
  \item {
   訓練模型時使用的輸入為一段長約三秒的聲音檔,使用PCM編碼,但sample rate, bit depth, channel未統一
  }
  \end{itemize}
\begin{figure}[H] %H为当前位置,!htb为忽略美学标准,htbp为浮动图形
\includegraphics[width=0.9\textwidth]{pic7} %插入图片,[]中设置图片大小,{}中是图片文件名
\caption{格式/長度各異的音訊} %標題
\end{figure}
\end{frame}
\begin{frame}{Methodology}{Output of model }
  \begin{itemize}
  \item {
   輸出為定義的十種label之一
  }
\item{警笛,喇叭,引擎怠速,槍響,狗吠,兒童嬉戲,音樂,施工,路面鑽鑿,冷氣運轉}
  \end{itemize}




\end{frame}
\begin{frame}{Methodology}{Each layer of model }
  \begin{itemize}
  \item {
    模型架構: 先將聲音檔案轉為mel-scaled spectrogram resize至512px512px的圖片\\讀入後使用兩層2D Convolution與Pooling\\Flatten後使用兩層densely-connected NN將節點收斂至10輸出
  }
  \end{itemize}
\begin{figure}[H] %H为当前位置,!htb为忽略美学标准,htbp为浮动图形
\includegraphics[width=0.9\textwidth]{pic0} %插入图片,[]中设置图片大小,{}中是图片文件名
\caption{轉為頻譜圖的音訊} %標題
\end{figure}
\end{frame}

\begin{frame}
\begin{figure}[H] %H为当前位置,!htb为忽略美学标准,htbp为浮动图形
\includegraphics[width=0.9\textwidth]{pic4} %插入图片,[]中设置图片大小,{}中是图片文件名
\caption{視覺化模型} %標題
\end{figure}
\end{frame}
\begin{frame}{Methodology}{How save model? }
  \begin{itemize}
  \item {
    訓練完成後使用.save()保存model
  }
  \end{itemize}
\end{frame}
\begin{frame}{Methodology}{File size of model }
  \begin{itemize}
  \item {
    儲存的model大小約281MB
  }
  \end{itemize}
\end{frame}
\begin{frame}{Methodology}{What’s loss functions, and why? }
  \begin{itemize}
  \item {
   由於希望對應輸出的十個值中一個為1其他為0,loss function選擇使用categorical crossentropy分類交叉熵函式 
  }
  \end{itemize}
\end{frame}
\begin{frame}{Methodology}{What’s optimizer and the setting of hyperparameter? }
  \begin{itemize}
  \item {
    optimizer使用SGD
  }
\item{learning rate=0.01}
\item{batch size=20}
\item{epoch=10}
  \end{itemize}
\end{frame}
\section{Dataset}
\begin{frame}{Dataset}{size of dataset}
  \begin{itemize}
  \item {
    訓練資料集共有5435個音檔,大小約4GB
  }
\item{
測試資料集共有3297個音檔,大小約2.6GB
}
  \end{itemize}
\end{frame}
\begin{frame}{Dataset}{collect/build dataset}
  \begin{itemize}
  \item {
    取自kaggle的開放資料集Urban Sound Classification
  }
  \end{itemize}

\begin{figure}[H] %H为当前位置,!htb为忽略美学标准,htbp为浮动图形
\includegraphics[width=0.9\textwidth]{pic5} %插入图片,[]中设置图片大小,{}中是图片文件名
\caption{資料集來源} %標題
\end{figure}


\end{frame}
\begin{frame}{Dataset}
 \begin{itemize}
  \item {
訓練時使用80\%分割
  }
  \end{itemize}
  \begin{itemize}
  \item {
    training samples  
 	\begin{itemize}
    	\item
     	4348筆training sample
   	 \end{itemize}
  }
  \end{itemize}
 \begin{itemize}
  \item {
   validating samples 
 	\begin{itemize}
    	\item
     	 1087筆validating sample
   	 \end{itemize}
  }
  \end{itemize}
 \begin{itemize}
  \item {
    testing samples 
 	\begin{itemize}
    	\item
     	 3297筆無label的testing sample
   	 \end{itemize}
  }
  \end{itemize}

\end{frame}

\section{Experimental Evaluation}
\begin{frame}{Experimental Evaluation}{Experimental environment}
  \begin{itemize}
  \item {
    training時使用kaggle kernal,硬體配置如下
\begin{itemize}
  \item {
    CPU: Intel(R) Xeon(R) CPU @ 2.30GHz (2 cores)
  }
  \item {
    RAM: 12.75GB avaliable
  }
  \item {
    GPU: Nvidia Tesla P100-PCIE-16GB (single core) 
  }
 \item {
    CUDA: Version 10.1
  }
  \end{itemize}
  }
  \end{itemize}

\begin{figure}[H] %H为当前位置,!htb为忽略美学标准,htbp为浮动图形
\includegraphics[width=0.7\textwidth]{pic6} %插入图片,[]中设置图片大小,{}中是图片文件名
\caption{硬體規格} %標題
\end{figure}
\end{frame}
\begin{frame}{Experimental Evaluation}{How many epochs set for training}
  \begin{itemize}
  \item {
    10個epoch後就有過擬合趨勢
  }
  \end{itemize}
\begin{figure}[H] %H为当前位置,!htb为忽略美学标准,htbp为浮动图形
\includegraphics[width=0.9\textwidth]{pic2} %插入图片,[]中设置图片大小,{}中是图片文件名
\caption{準確率隨epoch的變化} %標題
\end{figure}
\end{frame}
\begin{frame}{Experimental Evaluation}{Qualitative \& Quantitative evaluation}
  \begin{itemize}
  \item {
    目前模型準確率最高到0.86 
另外由於使用者任意輸入的音檔可能不在原先10個class裡面,設置當所有class的吻合度皆小於閥值時輸出unknown
  }
  \end{itemize}

\begin{figure}[H] %H为当前位置,!htb为忽略美学标准,htbp为浮动图形
\includegraphics[width=0.9\textwidth]{pic3} %插入图片,[]中设置图片大小,{}中是图片文件名
\caption{最高準確率} %標題
\end{figure}
\end{frame}
\begin{frame}{謝謝聆聽}
\begin{figure}[H] %H为当前位置,!htb为忽略美学标准,htbp为浮动图形
\centering
\includegraphics[width=1.4\textwidth]{pic8} %插入图片,[]中设置图片大小,{}中是图片文件名
\caption{最高準確率} %標題
\end{figure}
\end{frame}
\end{CJK}
\end{document}
